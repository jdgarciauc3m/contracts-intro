\section{Contract checking}

\begin{frame}[t,fragile]{Assertion level}
\begin{itemize}
  \item Every contract expression has an associated \emph{assertion level}.

  \vfill\pause
  \item Contract levels:
    \cppkey{default},
    \cppkey{audit},
    \cppkey{axiom}.
    \begin{itemize}
      \item Checks will be effectively performed depending on build mode.
    \end{itemize}

  \vfill\pause
  \item Default level can be omitted.

\begin{lstlisting}
void f(element & x) [[expects: x.valid()]];
void g(element & x) [[expects default: x.valid()]];
\end{lstlisting}

  \vfill\pause
  \begin{itemize}
    \item Cost of \cppkey{default} checking is expected to be small compared to function execution.
  \end{itemize}
\end{itemize}
\end{frame}

\begin{frame}[t,fragile]{Audit checks}
\begin{itemize}
  \item An \cppkey{audit} \emph{assertion level} is expected to be used in cases
where the cost of a run-time check is assumed to be large compared to function
execution.
    \begin{itemize}
      \item Or at least significant.
    \end{itemize}
\end{itemize}
\begin{lstlisting}
template <typename It, typename T>
bool binary_search(It first, It last, const T & x)
  [[expects audit: is_sorted(first,last)]];
\end{lstlisting}
\end{frame}

\begin{frame}[t,fragile]{Axiom checks}
\begin{itemize}
  \item An \cppkey{axiom} \emph{assertion level} is expected to be used in cases
where the run-time check will \textmark{never} be performed.
    \begin{itemize}
      \item Still they need to be valid C++.
      \item They are formal comments for humans and/or static analyzers.
    \end{itemize}
\end{itemize}

\vfill\pause
\begin{lstlisting}
template <typename InputIterator>
InputIterator my_algorithm(InputIterator first, InputIterator last)
  [[expects axiom: first!=last && reachable(first,last)]];
\end{lstlisting}

\vfill\pause
\begin{itemize}
  \item Axioms are not evaluated.
    \begin{itemize}
      \item They may contain calls to declared but undefined functions.
    \end{itemize}
\end{itemize}
\end{frame}

\begin{frame}[t]{Build levels}
\begin{itemize}
  \item Every translation is performed in a \emph{build level}:
    \begin{itemize}
      \item \textgood{off}: No run-time checking is performed.
      \item \textgood{default}: Checks with \cppkey{default} levels are checked.
      \item \textgood{audit}: Checks with \cppkey{default} and \cppkey{audit}
levels are checked.
    \end{itemize}
  
  \vfill\pause
  \item How do you select the \emph{build level}:
    \begin{itemize}
      \item \textbad{No way of selecting in source code.}
      \item An option from your compiler.
    \end{itemize}
\end{itemize}
\end{frame}

\begin{frame}[t,fragile]{Contract checking}
\begin{itemize}
\item If a function has multiple preconditions or postconditions
that would be checked, their \textmark{evaluation will be performed
in the order they appear}
\end{itemize}

\vfill\pause
\begin{lstlisting}
void f(int * p)
  [[expects: p!=nullptr]]
  [[expects: *p == 0]] // Only checked when p!=nullptr
{
  *p = 1;
}
\end{lstlisting}
\end{frame}

\begin{frame}[t,fragile]{Contract violation handlers}
\begin{itemize}
  \item A translation unit has an associated contract violation handler.

  \vfill\pause
  \item A contract violation handler is the function to be called when a
contract is broken.
    \begin{itemize}
      \item Function with specific signature.
\begin{lstlisting}
void (const std::contract_violation &);
\end{lstlisting}
    \end{itemize}

  \vfill\pause
  \item If you do not supply a handler, the default is \cppid{std::abort()}.

  \vfill\pause
  \item If you want to supply a handler:
    \pause
    \begin{itemize}[<+->]
      \item \textbad{No way of setting through source code}.
      \item \textbad{No way of asking which is current handler}.
      \item \textgood{An option in your compiler to supply it}.
      \item \textmark{Security sensitive systems may prevent arbitrary
handlers}.
    \end{itemize}

\end{itemize}
\end{frame}

\begin{frame}[t,fragile]{Information for the handler}
\begin{itemize}
  \item Function with specific signature.
\begin{lstlisting}
void (const std::contract_violation &);
\end{lstlisting}

  \vfill\pause
  \item Minimum information inf \cppid{contract\_violation}:
\begin{lstlisting}
  class contract_violation {
  public:
    uint_leaset32_t line_number() const noexcept;
    string_view file_name() const noexcept;
    string_view function_name() const noexcept;
    string_view comment() const noexcept;
    string_view assertion_level() const noexcept;
  };
\end{lstlisting}

  \vfill\pause
  \item Might get simplified by \cppid{std::experimental::source\_location}.
\end{itemize}
\end{frame}

\begin{frame}[t]{What is the violation source location?}
\begin{itemize}
  \item \textgood{Precondition}: Reported location is implementation defined.
    \begin{itemize}
      \item Implementations are encouraged but not required to report caller site.
    \end{itemize}
  \item \textgood{Postcondition}: Reported location is the callee site.
  \item \textgood{Assertion}: Location of the assertion itself.
  \vfill
  \item There is an orthogonal proposal to provide access to stack traces.
\end{itemize}
\end{frame}

\begin{frame}[t]{What happens after the violation handler?}
\begin{itemize}
  \item Two basic options:
    \begin{itemize}
      \item Program \emph{finishes} execution.
      \item Program resumes execution.
    \end{itemize}

  \vfill\pause
  \item An option in your compiler to select \emph{continuation mode}:
    \begin{itemize}
      \item \textmark{off}: Do not resume execution.
        \begin{itemize}
          \item Default option.
        \end{itemize}
      \item \textmark{on}: Resume execution.
    \end{itemize}

  \vfill\pause
  \item But remember:
    \pause
    \begin{itemize}[<+->]
      \item \textbad{No way of setting through source code}.
      \item \textbad{No way of asking which is current mode}.
    \end{itemize}
\end{itemize}
\end{frame}

\begin{frame}[t]{Why do we want to continue?}
\begin{itemize}
  \item Gradual introduction of contracts.

  \vfill\pause
  \item Testing the contracts themselves.

  \vfill\pause
  \item Plugin management.
\end{itemize}
\end{frame}

\begin{frame}[t,fragile]{Continuation mode and optimizations}
\begin{itemize}
  \item Assertion information may be used by optimizers.
\end{itemize}
\begin{lstlisting}
[[assert: ptr!=nullptr]];
//...
if (ptr==nullptr) { // Optimization opportunity
  do_stuff();
}
\end{lstlisting}
\vfill\pause
\begin{itemize}[<+->]
  \item If continuation mode is \textgood{off}, then \cppkey{if} is never reached.
  \item If continuation mode is \textgood{on}, then \cppkey{if} would be
reached.
    \begin{itemize}[<+->]
      \item But the \cppkey{if} might get \textbad{optimized out}!
      \item \textmark{Continuation after violation is technically undefined behavior}.
    \end{itemize}
\end{itemize}
\end{frame}

\begin{frame}[t,fragile]{Contracts and \textbf{noexcept}}
\begin{itemize}
  \item What happens to \cppkey{noexcept} function if its contract is broken?
    \begin{itemize}
      \item With continuation mode set to \cppkey{off} program finishes.
      \item With continuation mode set to \cppkey{on} program resumes.
      \item But, what if the handler throws an exception'
        \begin{itemize}
          \item Program invokes \cppid{terminate()} \emph{as-if} an exception was thrown inside
functions.
        \end{itemize}
    \end{itemize}
\end{itemize}
\begin{lstlisting}
void f(int x) [[expects: x > 0]];

void g() {
  f(-1); // Invokes terminate if handler throws
}
\end{lstlisting}
\end{frame}
