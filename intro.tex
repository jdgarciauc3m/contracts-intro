\section{Introduction}

\begin{frame}[t]{Correctness and Robustness}
\begin{itemize}
  \item In the design of a library there is a tension between two related
properties: \textmark{robustness} and \textmark{correctness}.

  \vfill
  \begin{itemize}

    \pause
    \item \textgood{Correctness} $\rightarrow$ Degree to which a software
component matches its specification.

    \pause
    \item \textgood{Robustness} $\rightarrow$ Ability of a software component to
react appropriately to abnormal conditions.

  \end{itemize}

  \vfill
  \pause
  \item Today many libraries use a single feature for managing both properties:
        \textgood{exception handling}.
    \begin{itemize}
      \item \textbad{We need to revisit this approach!}
    \end{itemize}

\end{itemize}
\end{frame}

\begin{frame}[t]{Exceptions in use}
\begin{itemize}
  \item When a failure happens, we use exceptions as an error reporting
mechanism.
    \begin{itemize}
      \item Notify that an error has occurred and needs to be handled.
      \item We decouple error identification from error handling.
      \item \textenum{Example}: Throwing \cppid{bad\_alloc}.
    \end{itemize}
  \vfill\pause
  \item When library detects an assumption was not met, it needs a mechanism to
react.
    \begin{itemize}
      \item Assumption not met $\Rightarrow$ \textmark{contract violation}.
      \item What do we do on contract violations?
        \begin{itemize}
          \item Ignore reality $\rightarrow$ \textbad{Do not call me!}
          \item Document.
          \item Throw exceptions.
        \end{itemize}
    \end{itemize}
  \vfill\pause
  \item \textmark{Robustness} and \textmark{correctness} are orthogonal
properties and \textgood{should be managed independently}.
\end{itemize}
\end{frame}

\begin{frame}[t]{Robustness in the C++ standard library}
\begin{itemize}
  \item \textgood{Robustness}: Identification and handling of abnormal
situations.
    \begin{itemize}
      \item Those situations occur in completely correct programs.
      \item \textmark{Example}: Failure to allocate memory.
      \item Is end of file a robustness issue?
    \end{itemize}

  \vfill\pause
  \item The C++ standard library identifies those cases by specifying
    \begin{enumerate}[i]
      \item the condition firing the situation.
      \item the exception that will be thrown to notify.
    \end{enumerate}
  \vfill\pause
  \item \cppid{T * allocator<T>::allocate(std::size\_t n);}
\fcolorbox{black}{blue!20!white}{
\emph{Throws}: \cppid{bad\_alloc} if storage cannot be obtained.
}
\end{itemize}
\end{frame}

\begin{frame}[t]{Correctness and contracts}
\begin{itemize}
  \item \textgood{Correctness} $\rightarrow$ Finding programming errors.
    \begin{itemize}
      \item Yes! Sometimes we write incorrect software.
    \end{itemize}

  \vfill\pause
  \item Who's guilty?
  \item A contract violation happens because:
    \begin{itemize}
      \item A caller does not fulfil the expectations before calling a function.
      \item A callee does not fulfill what should be ensured after its own
execution.
    \end{itemize}
  \vfill\pause
  \item A key difference:
    \begin{itemize}
      \item A program failure is usually due to external conditions and cannot
be avoided.
      \item A contract violation \textmark{\emph{should}} never happen in a
correct program.
    \end{itemize}
\end{itemize}
\end{frame}

\begin{frame}[t]{Correctness in the C++ standard library}
\begin{itemize}
  \item From the standard:
\end{itemize}
\vspace{.25ex}
\fcolorbox{black}{blue!30!white}{
\begin{minipage}{\textwidth}
Violation of the preconditions specified in a function's \emph{Requires}: paragraph
results in undefined behavior unless the functions \emph{Throws}: paragraph specifies
throwing an exception when the precondition is violated.
\end{minipage}
}
  \vfill\pause
\begin{itemize}
  \item In practice, there are two approaches in the standard library:
    \begin{itemize}
      \item Do nothing $\rightarrow$ \textbad{Undefined behaviour}.
      \item Notify $\rightarrow$ \textmark{Throw an exception}.
    \end{itemize}
\end{itemize}
\end{frame}
