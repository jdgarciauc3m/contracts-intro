\section{Contracts in C++}

\begin{frame}[t]{What is a contract?}
\begin{itemize}
  \item A contract is the set of \textmark{preconditions},
\textmark{postconditions} and \textmark{assertions} associated to a function.
    \begin{itemize}
      \vfill\pause
      \item \textgood{Precondition}: What are the \emph{expectations} of the function?
      \vfill\pause
      \item \textgood{Postconditions}: What must the function \emph{ensure}
            upon termination?
      \vfill\pause
      \item \textgood{Assertions}: What predicates must be satisfied in specific
            locations of a function body?
    \end{itemize}
\end{itemize}
\end{frame}

\begin{frame}[t,fragile]{Expectations}
\begin{itemize}
  \item \textgood{Precondition}
    \begin{itemize}
      \item A predicate that should hold upon entry into a function.
      \item It expresses a function's expectation on its arguments and/or the
            state of objects that may be used by the function.
      \item Expressed by attribute \cppkey{expects}.
    \end{itemize}
\end{itemize}

\vfill\pause
\begin{lstlisting}
double sqrt(double x) [[expects: x>0]];
\end{lstlisting}
\pause
\begin{lstlisting}
class queue {
  //...
  void push(const T & x) [[expects: !full()]];
  //...
};
\end{lstlisting}

\vfill\pause
\begin{itemize}
  \item Preconditions use a modified attribute syntax.
  \item The expectation is part of the function declaration.
\end{itemize}

\end{frame}

\begin{frame}[t,fragile]{Assurances}
\begin{itemize}
  \item \textgood{Postcondition}
    \begin{itemize}
      \item A predicate that should hold upon exit from a function.
      \item It expresses the conditions that a function should ensure for the 
            return value and/or the state of objects that may be used by the 
            function.
      \item Postconditions are expressed by \cppkey{ensures} attributes.
    \end{itemize}
\end{itemize}

\vfill\pause
\begin{lstlisting}
double sqrt(double x)
  [[expects: x>=0]]
  [[ensures result: result>=0]];
\end{lstlisting}

\vfill\pause
\begin{itemize}
  \item Postconditions may introduce a name for the result of the function.
\end{itemize}
\end{frame}

\begin{frame}[t,fragile]{Assertions}
\begin{itemize}
  \item \textgood{Assertions}
    \begin{itemize}
      \item A predicate that should hold at its point in a function body.
      \item It expresses the conditions that must be satisfied, on objects that
are accessible at its point in a body.
      \item Assertions are expressed by assert attributes.
    \end{itemize}
\end{itemize}

\vfill\pause
\begin{lstlisting}
double add_distances(const std::vector<double> & v) 
  [[expects r: r>=0.0]]
{
  double r = 0.0;
  for (auto x : v) {
    [[assert: x >= 0.0]];
    r += x;
  }
  return r;
}
\end{lstlisting}
\end{frame}

\begin{frame}[t]{Effect of contracts}
\begin{itemize}
  \item A contract has no observable effect on a correct program (except
performance).
    \begin{itemize}
      \item The only semantic effect of a contract happens if it is violated.
    \end{itemize}

  \vfill\pause
  \item Why do we use attributes syntax?
    \begin{itemize}
      \item Contract may be checked or not.
      \item Attributes are not part of function type.
      \item However, \textbad{contracts are not an optional feature}.
    \end{itemize}

  \vfill\pause
  \item Contracts checking and corresponding effects depend on build system
settings.
    \begin{itemize}
      \item Default: Contract violation $\Rightarrow$ Program
termination.
    \end{itemize}
\end{itemize}
\end{frame}

\begin{frame}[t,fragile]{Repeating a contract}
\begin{itemize}
  \item Any redeclaration of a function has either the same contract or
completely omits the contract.
\end{itemize}
\vfill\pause
\begin{lstlisting}
int f(int x)
  [[expects: x>0]]
  [[ensures r: r >0]];

int f (int x) ; // OK. No contract.

int f ( int x)
  [[expects: x>=0]]; // Error missing ensures and different expects

int f(int x)
  [[expects: x>0]]
  [[ensures r: r >0]]; //OK. Same contract.
\end{lstlisting}
\end{frame}

\begin{frame}[t,fragile]{Repeating a contract}
\begin{itemize}
  \item But argument names may differ.
\end{itemize}

\vfill\pause
\begin{lstlisting}
int f(int x)
  [[expects: x>0]]
  [[ensures r: r >0]];

int f(int y)
  [[expects: y>0]]
  [[ensures z: z >0]];


\end{lstlisting}
\end{frame}

